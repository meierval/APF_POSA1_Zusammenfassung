\chapter{Microkernel}

\section{Summary}
Das Micokernel Pattern wird bei Software Systemen angewendet, die sich an ändernde System Anforderungen anpassen können müssen. Es separiert einen aufs Minimum reduzierten Kern von zusätzlichen Funktionalitäten und Kunden spezifischen Teilen. Der Microkernel dient auch als Socket für diese Erweiterungen und er koordiniert ihre Zusammenarbeit.
\section{Context}
Entwicklung von mehreren Applikationen welche ähnliche Interfaces verwenden, welche auf der selben Grundfunktionalität aufbauen.
\section{Problem}
Software für ein Anwendungsgebiet zu entwickeln, welches mit einem breiten Spektrum von ähnlichen Standards und Technologien umgehen muss ist nicht einfach. Oft wird noch eine lange Lebenszeit gefordert, in welcher neue Technologien aufkommen und alte sich verändern. In solchen Fällen schafft das Microkernel Pattern Abhilfe. Das typische Beispiel für solche Software sind Betriebssysteme. 

\section{Solution}

\subsection{Structure}

\section{Consequences}
\begin{itemize}
    \pro{bla}
    \con{bla}
\end{itemize}

\section{Known Uses}
\begin{itemize}
	\item item1
	\item item2
\end{itemize}

\section{Relationships}
\begin{itemize}
	\item \textit{bla} 
\end{itemize}

\section{Exam Questions}
\begin{itemize}
  \item Behauptung: dies ist eine Behauptung? (Lösung)
    \item Frage: Dies ist eine Frage? (Lösung)
\end{itemize}
