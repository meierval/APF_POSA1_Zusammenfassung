\chapter{Presentation-Abstraction-Control}

\section{Summary}
Das Presentation-Abstraction-Control Pattern definiert eine Struktur für interaktive Software Systeme in Form einer Hierarchie von zusammenarbeitenden Agents. Jeder Agent ist dabei für einen spezifischen Teil der Funktionalität der Software zuständig. Ein Agent besteht aus den Komponenten Presentation, Abstraction und Control. Diese Unterteilung separiert den Mensch-Computer-Interaktions Aspekt des Agents von seinem funktionellen Kern und der Kommunikation mit anderen Agents.
\section{Context}
Entwicklung einer Interaktiven Applikation mit Einsatz von Agents.
\section{Problem}
Interaktive Systeme können oft als eine Sammlung von zusammenarbeitenden Agents gesehen werden. In solch einem System ist jeder Agent auf eine bestimmte Aufgabe spezialisiert. 

\section{Solution}


\subsection{Structure}

\section{Consequences}
\begin{itemize}
    \pro{bla}
	\con{bla}
\end{itemize}

\section{Known Uses}
\begin{itemize}
	\item Blabla
\end{itemize}

\section{Relationships}
\begin{itemize}
	\item \textit{Keine} 
\end{itemize}

\section{Exam Questions}
\begin{itemize}
  \item Behauptung: dies ist eine Behauptung? (Lösung)
    \item Frage: Dies ist eine Frage? (Lösung)
\end{itemize}
