\chapter{Presentation-Abstraction-Control}

\section{Summary}
Das Presentation-Abstraction-Control (PAC) Pattern definiert eine Struktur für interaktive Software Systeme in Form einer Hierarchie von zusammenarbeitenden Agents. Jeder Agent ist dabei für einen spezifischen Teil der Funktionalität der Software zuständig. Ein Agent besteht aus den Komponenten Presentation, Abstraction und Control. Diese Unterteilung separiert den Mensch-Computer-Interaktions Aspekt des Agents von seinem funktionellen Kern und der Kommunikation mit anderen Agents.
\section{Context}
Entwicklung einer Interaktiven Applikation mit Einsatz von Agents.
\section{Problem}
Interaktive Systeme können oft als eine Sammlung von zusammenarbeitenden Agents gesehen werden. In solch einem System ist jeder Agent auf eine bestimmte Aufgabe spezialisiert, und alle Agents zusammen bilden das System. Folgende Forces haben einen Einfluss auf die Lösung:
\begin{itemize}
	\item Agents haben oft ihren eigenen State bzw. ihre eigenen Daten. Um zusammenarbeiten zu können brauchen die Agents einen Mechanismus zum Austausch von Daten, Messages und Events.
	\item Interaktive Agents bieten oft ihr eigenes Interface.
	\item Systeme entwickeln sich über die Zeit. Änderungen an einzelnen Agents oder das hinzufügen solcher sollte keinen Effekt auf das gesamte System haben.
\end{itemize}
\section{Solution}
Strukturierung der interaktiven Applikation als eine baumartige Hirarchie von PAC Agents, wobei es einen Top-Level Agent, mehrere Intermediate-Level Agents und noch mehr Bottom-Level Agents gibt. Jeder Agent ist abhängig von allen Agents auf einem höheren Level rauf bis zum Top-Level Agent. Im folgenden die Aufgaben der Level:
\paragraph{Top-Level} Funktioneller Kern des Systems
\paragraph{Intermediate-Level}
\paragraph{Bottom-Level}
Ein Agent ist in die Komponenten Presentation, Abstraction und Control unterteilt. Die Presentation Komponente bietet das sichtbare Verhalten des PAC Agent. Von der Abstraction Komponente wird das Datenmodell verwaltet, sie bietet Funktionalität welche auf diesen Daten operiert. Die Control Komponente verbindet Presentation und Abstraction, und bietet zusätzlich Funktionalität die dem Agent ermöglicht mit anderen Agents zu kommunizieren.


\subsection{Structure}

\section{Consequences}
\begin{itemize}
    \pro{bla}
	\con{bla}
\end{itemize}

\section{Known Uses}
\begin{itemize}
	\item Blabla
\end{itemize}

\section{Relationships}
\begin{itemize}
	\item \textit{Keine} 
\end{itemize}

\section{Exam Questions}
\begin{itemize}
  \item Behauptung: dies ist eine Behauptung? (Lösung)
    \item Frage: Dies ist eine Frage? (Lösung)
\end{itemize}
